\chapter{Vorauslegung}
\label{chap:Vorauslegung}
% Ziel hier: den Leser in die Lage versetzen, die wesentliche Struktur der Arbeit nachvollziehen zu können (Vorgehen / Geräte / Materialien / Verfahren [also auch Software]  / getroffene Annahmen ... sollten hier erklärt werden)
Die Flugdaten kommen aus einer Trajektoriensimulation aus dem Simulationsprogramm OpenRocket, welche vom Triebwerk-Subsystem durchgeführt wurde.
Diese Flugdaten~(\ref{fig:flugdaten_trajektoriensimulation}) bilden eine Maximalabschätzung der Aerodynamischen Aufheizung und Flugdauer durch
maximale Schubkraft und Dauer mit \SI{8}{\kilo\newton} für \SI{43}{\second}, die von \ac{blast} erreicht werden können.

\section{Anforderungen}

Da die Kühlung zeitgleich zu der Avionik entwickelt wurde, musste auf eine genaue Analyse aller Komponenten der Avionik verzichtet werden.
Stattdessen wurde anhand des bereits festgelegten Microcontrollers STM32H743ZGT6, der auf den redundanten Flugcomputern verwendet wird,
die Auslegung durchgeführt.
Aus dem Datenblatt des Microcontrollers folg eine maximale Sperrschichttemperatur von $T_\text{J} = \SI{125}{\degreeCelsius}$~\cite{STM32}
und ein Sperrschicht-Gehäuse Wärmeleitwiederstand von $\Theta_\text{JC} = \SI{23.9}{\degreeCelsius\per\watt}$~\cite{STM32}. Mit einem konservativen
Sicherheitsfaktor von 1.5, um bisher unbekannte Bauteile zu berücksichtigen, folgt daraus $\Theta_\text{JC,safety} = \SI{35.85}{\degreeCelsius\per\watt}$
und eine maximale Gehäusetemperatur von $T_\text{C} = \SI{89.15}{\degreeCelsius}$. Im Kontext der Elektronik ist mit Gehäuse immer die
Oberseite der elektronischen Komponente gemeint.
Die Kühlung soll außerdem eine hohe Zuverlässigkeit haben, welche durch Verwendung von ausschließlich passiven Bauteilen gewehrleistet wird.
Dadurch kann aufwendiges und teures testen und verifizieren von aktiven Bauteilen mit mechanischer oder elektrischer Funktion vermieden werden und es besteht bei
nicht nominalen Flügen eine geringere Ausfallwahrscheinlichkeit durch die inherent größeren Toleranzen passiver Bauteile.

Dem Energieerhaltungssatz nach haben der \ac{fcc}, die Kameras und weitere Elektronik die keine Leistung abgibt, gegenüber etwa
der \ac{pcdu} und Funkplatine welche Leistung in Form von Strom und elektromagnetischer Strahlung abgeben, einen Wirkungsgrad von
\SI{0}{\percent}, da Logikoperationen physikalisch gesehen keine Arbeit sind. Resultierend wird der komplette Stromverbrauch
in Wärme umgewandelt.

\begin{table}
  \centering
  \caption{Leistung der Avionik}\label{tab:avionik_leistung}

  \begin{tabular}{lp{4cm}ll}
    \toprule[1pt]
    Komponente & Spannung \& Strom & Wirkungsgrad & Wärmestrom \\
    \midrule[0.5pt]

    STM32H743ZGT6 &
      \mbox{$V_\text{DD}=\SI{3.3}{\volt}$},\newline
      $I_\text{DD}=\SI{536}{\milli\ampere}$~\cite{STM32} &
      $\approx \SI{0}{\percent}$ & \SI{1.769}{\watt} \\
    $\dot{Q}_\text{ges}$ & & & \SI{7.075}{\watt}\\

    \midrule[0.5pt]
    RunCam Split 4 V2 &
      \mbox{$V_\text{DD}=\SI{5}{\volt}$},\newline
      $I_\text{DD}=\SI{450}{\milli\ampere}$~\cite{RunCam-Split4V2} &
      $\approx \SI{0}{\percent}$ & \SI{2.25}{\watt} \\
    $\dot{Q}_\text{ges}$ & & & \SI{9}{\watt}\\

    \midrule[0.5pt]
    Thebe-II &
      \mbox{$V_\text{DD}=\SI{3.6}{\volt}$},\newline
      $I_\text{DD}=\SI{500}{\milli\ampere}$~\cite{WE-ThebeII-UM-2024} &
      $\approx \SI{30}{\percent}$~\cite{WE-ThebeII-UM-2024} & \SI{1.3}{\watt} \\

    \midrule[0.5pt]
    \ac{pcdu} & & $\approx \SI{30}{\percent}$ & \SI{9.3}{\watt} \\

    \midrule[0.5pt]
    \midrule[0.5pt]
    $\dot{Q}_\text{ges, safety}$ & & & \SI{40}{\watt} \\

    \bottomrule[1pt]
  \end{tabular}
\end{table}

Die Leistung der Avionik in~\ref{tab:avionik_leistung} ergibt sich durch den Maximalverbrauch der \ac{fcc} mikrocontroller
(STM32H743ZGT6) bei maximaler clock rate (\SI{400}{\mega\hertz}) und vollständig aktiver Peripherie, der Kameras und einer
Abschätzung der restlichen Komponenten ohne Quellenangabe. Der aus \ref{tab:avionik_leistung} resultierende gesamten Wärmestrom
der Avionik mit \SI{40}{\watt} ist mit einem gewöhnlichen Laptop vergleichbar.

\begin{figure}
    \centering

    % Column 1, Row 1
    \begin{subfigure}{0.48\textwidth}
        \centering
        \includegraphics[width=\linewidth]{../../Code/acceleration_over_time.pdf}
        \caption{Beschleunigung während Flug}
        \label{fig:acceleration_over_time}
    \end{subfigure}
    \hfill
    % Column 2, Row 1
    \begin{subfigure}{0.48\textwidth}
        \centering
        \includegraphics[width=\linewidth]{../../Code/altitude_over_time.pdf}
        \caption{Flughöhe}
        \label{fig:altitude_over_time}
    \end{subfigure}

    \vspace{1em}

    % Column 1, Row 2
    \begin{subfigure}{0.48\textwidth}
        \centering
        \includegraphics[width=\linewidth]{../../Code/pressure_over_time.pdf}
        \caption{Statischer Luftdruck während Flug}
        \label{fig:pressure_over_time}
    \end{subfigure}
    \hfill
    % Column 2, Row 2
    \begin{subfigure}{0.48\textwidth}
        \centering
        \includegraphics[width=\linewidth]{../../Code/temperature_over_time.pdf}
        \caption{Statische Lufttemperatur während Flug}
        \label{fig:temperature_over_time}
    \end{subfigure}

    \vspace{1em}

    % Column 1, Row 3
    \begin{subfigure}{0.48\textwidth}
        \centering
        \includegraphics[width=\linewidth]{../../Code/velocity_over_time.pdf}
        \caption{Geschwindigkeit während Flug}
        \label{fig:velocity_over_time}
    \end{subfigure}
    \hfill
    % Column 2, Row 3
    \begin{subfigure}{0.48\textwidth}
        \centering
        \includegraphics[width=\linewidth]{../../Code/dynp_during_flight.pdf}
        \caption{Dynamischer Druck während Flug}
        \label{fig:dynp_over_time}
    \end{subfigure}

    \caption{Flugdaten der Trajektoriensimulation}\label{fig:flugdaten_trajektoriensimulation}
\end{figure}

\section{Thermale Schnittstelle}\label{sec:thermale_schnittstelle}

Um mit der Abwärme der Avionik umgehen zu können, muss sie effektiv gesammelt und abtransportiert werden.
Oft werden in der Luft- und Raumfahrtindustrie Kühlkreisläufe mit einem Arbeitsfluid verwendet. Diese benötigen jedoch
meist bewegliche Bauteile wie Pumpen, welche die Ausfallwahrscheinlichkeit erhöhen. Alternativ gibt es auch
Möglichkeiten durch erzwungene Konvektion ein Arbeitsfluid anzutreiben oder Materialien mit hoher Wärmeleitfähigkeit
zu verwenden. Beide Methoden bieten in Kombination eine günstige Integrierbarkeit und geringen Wärmeleitwiederstand,
ohne Bewegliche Teile zu verwenden.

Das Thermale Interface wird auf Systemebene analysiert, da eine Entwicklung auf \ac{pcb} Ebene wie bereits erläutert nicht
möglich ist, ohne vollständig entwickelte Elektronik.

\subsection{Heatpipes}\label{sec:waermerohre}

Heatpipes (Wärmerohre) sind eine Möglichkeit durch erzwungene Konvektion Wärme zu transportieren. Reguläre Heatpipes
sind vollständig geschlossene Rohre mit einer Flüssigkeit im inneren und einer Kapillarstruktur an der Innenwand,
so dass ein freier Kanal in der Mitte bleibt. Bei der Wärmequelle
verdampft die Flüssigkeit aus der Kapillarstruktur und bei der Wärmesenke kondensiert es wieder, wodurch der resultierende Massenstrom einen
Kreislauf bildet. Besonders effektiv sind Heatpipes durch die Nutzung der Verdampfungsenthalpie beim Flüssig-Gas Übergang an der Wärmequelle,
wodurch sehr hohe Wärmestromdichten erreicht werden können. Eine Schematische Darstellung eines Wärmerohrs kann in \ref{fig:waermerohr}
gesehen werden.

Eine Weiterentwicklung davon sind Loop Heatpipes die, wie der Namen bereits impliziert einen Kreislauf bilden, indem es eine
separate Flüssig- und Dampfleitung gibt, welche jeweils am Verdampfer bzw. Kondensator miteinander verbunden ist.
Besonders von Vorteil sind Loop Heatpipes, wenn größere Distanzen überbrückt werden müssen, oder eine relativ zuverlässige
Funktion unabhängig von Orientierung und Gravitation gebraucht wird. Aufgrund der erhöhten Komplexität von Loop Heatpipes, der Möglichkeit
die Orientierung frei zu bestimmen, den relativ geringen Distanzen innerhalb der Avionik-sektion und dem Mangel an Kommerziell erhältlichen
Loop Heatpipes wird eine reguläre Heatpipe gewählt.


\begin{figure}
  \centering
  \begin{tikzpicture}
    \draw[thick] (-5,0) arc [start angle=90, end angle=270, x radius=1.5cm, y radius= 1.5cm]; % Kappen
    \draw[thick] (5,0) arc [start angle=90, end angle=-90, x radius=1.5cm, y radius= 1.5cm];

    \draw[thick] (-5,0) -- (5,0); % hülle
    \draw[thick] (-5,-3) -- (5,-3);

    \draw[dashed] (-5,0) rectangle (5,-1); % Wick Abgrenzung
    \draw[dashed] (-5,-2) rectangle (5,-3);

    \draw[->, thick, -{Stealth[length=0.25cm]}] (-1,-1.75) -- node [midway, above] {Dampfstrom} (1,-1.75); % Dampf Pfeil
    \draw[->, thick, -{Stealth[length=0.25cm]}] (1,-2.75) -- node [midway, above] {Flüssigstrom}(-1,-2.75); % Flüssigkeitspfeil
    \draw[->, thick, -{Stealth[length=0.25cm]}] (1,-0.5) -- (-1,-0.5); % Flüssigkeitspfeil

    \node at (-4,-4) [style={single arrow, draw}, minimum height=0.5cm, minimum width=1.5cm, shape border rotate=90, thick]{$\dot{Q}$}; % Wärmestrom pfeil
    \node at (4,-3.75) [style={single arrow, draw}, minimum height=0.5cm, minimum width=1.5cm, shape border rotate=270, thick]{$\dot{Q}$}; % Wärmestrom pfeil
    
    \draw[->, thick, -{Stealth[length=0.25cm]}] (-4,0.25) node [above=1pt] {Kapillarstruktur} -- (-3,-0.5); % Kapillar Pfeil
    \draw[->, thick, -{Stealth[length=0.25cm]}] (-4,0.25) -- (-3,-2.5);
  \end{tikzpicture}
  \caption{Wärmerohr Aufbau und Funktionsweise}\label{fig:waermerohr}
\end{figure}

Ein wichtiger Aspekt von Heatpipes ist, dass der Wärmeleitwiederstand durch Biegungen und Anbindung von mehreren Quellen um bis zu \SI{100}{\percent}
steigen kann \cite{Mooney-2020}. Des weiteren hängt besonders bei regulären Heatpipes der Wärmeleitwiederstand von der effektiven Beschleunigung ab,
da die höhere Dichte der Flüssigphase eine beschleunigende Wirkung auf die Konvektion hat, wenn die Wärmequelle unten orientiert ist. Sollte die Heatpipe jedoch
\glqq überkopf \grqq{} arbeiten, sodass die Wärmequelle oben orientiert ist, muss die Konvektion gegen die Beschleunigung arbeiten und verliert Leistung bzw. hat
einen erhöhten Wärmeleitwiederstand.

Ausgewählt wurde die QG-SHP-D5-400MN Heatpipe von Quick-Ohm Küpper \& Co. GmbH aus Kupfer mit Mesh-Gewebe als Kapillarstruktur von \SI{400}{\milli\meter} Länge und
\SI{5}{\milli\meter} Durchmesser. Diese Heatpipe kann eine Leistung von \SI{40}{\watt} übertragen.

Weiterhin wird die Heatpipe als \ac{rom} mit einem einfachen Widerstand ersetzt, der dem Wärmeleitwiederstand von $R_\mathrm{Heatpipe} = \SI{0,3}{\kelvin\per\watt}$ der Heatpipe aus dem Datenblatt~\cite{QuickOhm-Heatpipe-5x400} entspricht.
Dadurch wird eine sehr komplexe Modellierung abhängig von Temperaturen, Biegungen, Ausrichtung, Beschleunigung und Anzahl an Wärmequellen sowie deren Leistung und Positionen vermieden.

\subsection{Wärmeleitbänder}\label{sec:waermebaender}

Um die \ac{pcb} mit dem Wärmerohr zu verbinden werden Wärmeleitbänder aus verschiedenen Materialien analysiert.
Wärmeleitbänder sind flexible Verbindungsteile mit hoher Wärmeleitfähigkeit die Wärmebrücken zwischen mehreren Bauteilen gewährleisten.
\ac{pgs} ist gegenüber herkömmlichen Materialien besonders interessant durch die extrem hohe Wärmeleitfähigkeit innerhalb der Ebene,
da diese der Ebene von der Molekülstruktur des Graphit entspricht. Außerdem ist es ein relativ flexibles Material, bei einer üblichen Dicke von $\approx \SIrange{10}{100}{\micro\meter}$.
Ein Nachteil von \ac{pgs} ist die im Kontrast zur Ebene sehr niedrige Wärmeleitfähigkeit durch die Ebene, infolge von wenigen
molekularen Brücken zwischen den Gitterstrukturen. Dementsprechend wird \ac{pgs} und andere Arten von Graphit Folien hauptsächlich zur
Wärmeverteilung auf der Oberfläche von Bauteilen verwendet um Wärmestromdichten zu verringern und homogenere Temperaturverteilungen zu erreichen.

Das effektive erhöhen des Querschnitts von \ac{pgs} durch Schichtung mehrerer Folien aufeinander ermöglicht es jedoch die hohe
Wärmeleitfähigkeit in der Ebene auch zum thermischen koppeln mehrerer Bauteile zu verwenden. Diese Anwendung hat besonders in der 
Raumfahrt durch ermöglichte Masseneinsparungen Halt gefunden. Eine Kommerzielle Reihe an solchen Wärmeleitbändern aus gängigen Materialien sieht man in~\ref{fig:thermalstraps_commercial}.
Der Tabelle~\ref{tab:strap_materials} nach ist \ac{pgs} das beste Kompromiss für die geforderten Eigenschaften. Um jedoch zu vermeiden, dass
bei starken Vibrationen aufgrund der Flexibilität des \ac{pgs} Kontakt mit der Elektronik und mögliche Kurzschlüsse entstehen, muss das Wärmeleitband mit einer elektrisch
isolierenden Ummantlung versehen werden.

\begin{figure}
  \centering
  \includegraphics[width=\textwidth]{thermal_straps_commercial.png}
  \caption{Kommerziell erhältliche Wärmeleitbänder aus \ac{pgs} (links), Kupfer und Aluminium~\cite{Thermal-Straps}}\label{fig:thermalstraps_commercial}
\end{figure}

\definecolor{good}{RGB}{200,255,200}   % hellgrün
\definecolor{medium}{RGB}{255,255,200} % hellgelb
\definecolor{bad}{RGB}{255,200,200}    % hellrot

\begin{table}
  \centering
  \caption{Ampelbewertung von Materialien für Wärmeleitbänder.}\label{tab:strap_materials}

  % 1. Spalte als Box-Spalte, damit \\ innerhalb der ZELLE umbricht
  \begin{tabular}{>{\raggedright\arraybackslash}m{3cm} m{3.2cm} m{3.2cm} m{3cm}}
    \toprule[1pt]
    Eigenschaft & Kupfer\cite{Thermtest-DB} & Aluminium\cite{Thermtest-DB} & PGS \nobreak{(Graphit)}\cite{HPMS-PGS} \\
    \midrule[0.5pt]

    \makecell[l]{Wärmeleit-\\fähigkeit\\in Ebene}
      & \cellcolor{medium}\SI{397.48}{\watt\per\meter\per\kelvin}
      & \cellcolor{bad}\SI{225.94}{\watt\per\meter\per\kelvin}
      & \cellcolor{good}\SIrange{1050}{1800}{\watt\per\meter\per\kelvin} \\

    \makecell[l]{Wärmeleit-\\fähigkeit\\durch Ebene}
      & \cellcolor{good}\SI{397.48}{\watt\per\meter\per\kelvin}
      & \cellcolor{medium}\SI{225.94}{\watt\per\meter\per\kelvin}
      & \cellcolor{bad}\SIrange{10}{26}{\watt\per\meter\per\kelvin} \\

    Dichte
      & \cellcolor{bad}\SI{8940}{\kilogram\per\cubic\meter}
      & \cellcolor{medium}\SI{2698}{\kilogram\per\cubic\meter}
      & \cellcolor{good}\SIrange{1500}{2100}{\kilogram\per\cubic\meter} \\

    Elektrische \makecell[l]{\\Isolation}
      & \cellcolor{bad}Schlecht
      & \cellcolor{bad}Schlecht
      & \cellcolor{bad}Schlecht \\
    \bottomrule[1pt]
  \end{tabular}
\end{table}

Aufgrund der höchsten Wärmeleitfähigkeit in der Ebene vom \ac{pgs} HGS-012 der Firma HPMS Graphite~\cite{HPMS-PGS} wurde dieses ausgewählte.
Um ein verwendbares Wärmeleitband zu konstruieren, soll dieses aus 32 Schichten bestehen, \SI{4}{\centi\meter} breit und 
\SI{10}{\centi\meter} lang sein, wodurch es ermöglicht werden soll, dass die Heatpipe keine Biegungen hat.
Die Anbindungen bzw. Endstücke der Wärmeleitbänder, sowie Kontaktwiederstände durch Klebstoffe oder ähnliche Verbindungsmethoden werden ignoriert.
Der Wärmeleitwiederstand ergibt sich durch einsetzen von Gleichung \ref{eq:fourier_1d} in \ref{eq:waermewiederstand}:

\begin{equation*}
  R_\mathrm{Wärmeleitband} = \frac{\Delta x}{\lambda A}
\end{equation*}

Mit $\Delta x = \SI{10}{\centi\meter}$, $\lambda = \SI{1800}{\watt\per\meter\kelvin}$ und $A = 32 \cdot \SI{0,012}{\milli\meter} \cdot \SI{4}{\centi\meter} = \SI{15,36}{\milli\meter\squared}$
ergibt sich $R_\mathrm{Wärmeleitband} = \SI{3,617}{\kelvin\per\watt}$

\subsection{Gesamte Schnittstelle}\label{sec:gesamte_schnittstelle}

Mittels einer Kombination von \ac{pgs} und Heatpipe kann eine leicht integrierbare Wärmebrücke gebildet werden, die den Wärmeleitwiederstand
minimiert. Eine Schematische Darstellung der Thermalen Schnittstelle ist in \ref{fig:thermale_schnittstelle} zu sehen.

Wenn angenommen wird, dass die Avionik aus vier separaten \ac{pcb} mit einer Gesamtleistung von \SI{40}{\watt}~(\ref{tab:avionik_leistung}) besteht, müssen pro Wärmeleitband \SI{10}{\watt} übertragen werden.
Dabei entsteht nach Gleichung \ref{eq:waermewiederstand} eine Temperaturerhöhung über das Wärmeleitband von $\SI{10}{\watt} \cdot \SI{3,617}{\watt\per\kelvin} = \Delta T_\mathrm{Wärmeleitband} = \SI{36,17}{\kelvin}$.
Die Heatpipe überträgt den vollständigen Wärmestrom und hat eine Temperaturerhöhung von $\SI{40}{\watt} \cdot \SI{0,3}{\watt\per\kelvin} = \Delta T_\mathrm{Heatpipe} = \SI{12}{\kelvin}$.

Von der Quelle bis zur Senke ergibt sich also ein Temperaturgradient von $\Delta T_\mathrm{Heatpipe} + \Delta T_\mathrm{Wärmeleitband} = \Delta T_\mathrm{ges} = \SI{48,17}{\kelvin}$.
Eine Schematische Darstellung der Schnittstelle sieht man in \ref{fig:thermale_schnittstelle}. Für die Nötige Temperatur an der Senke
erhält man $T_\mathrm{Senke} = T_C - \Delta T_\mathrm{ges} = \SI{314,13}{\kelvin}$.

\begin{figure}
  \centering
  \begin{tikzpicture}
    \node at (0,0) [style={single arrow, draw}, minimum height=0.5cm, minimum width=1.5cm, shape border rotate=90, thick]{$\dot{Q}$}; % Wärmestrom pfeil

    \draw[thick] (0,-1) -- (0,-3.1);

    \draw[thick] (0,-3.1) -- (-0.2,-3.2); % heatpipe resistor
    \draw[thick] (-0.2,-3.2) -- (0.2,-3.4);
    \draw[thick] (0.2,-3.4) -- (-0.2,-3.6);
    \draw[thick] (-0.2,-3.6) -- (0.2,-3.8);
    \draw[thick] (0.2,-3.8) -- (-0.2,-4);
    \draw[thick] (-0.2,-4) -- (0.2,-4.2);
    \draw[thick] (0.2,-4.2) -- (0,-4.3);

    \node[rotate=90] at (-0.5,-3.7) {Heatpipe};
    \node at (2.5,-3.7) {$R_{\mathrm{Heatpipe}} = \SI{0,3}{\kelvin\per\watt}$};

    \draw[thick] (0,-4.3) -- (0,-6.4);

    \draw[thick] (0,-6.4) -- (2.1,-6.4);

    \fill (0,-6.4) circle (1.5pt); %knotenpunkt

    \begin{scope}[shift={(-1,-6.4)}, rotate=90]
      \draw[thick] (0,-3.1) -- (-0.2,-3.2); % thermal strap resistor
      \draw[thick] (-0.2,-3.2) -- (0.2,-3.4);
      \draw[thick] (0.2,-3.4) -- (-0.2,-3.6);
      \draw[thick] (-0.2,-3.6) -- (0.2,-3.8);
      \draw[thick] (0.2,-3.8) -- (-0.2,-4);
      \draw[thick] (-0.2,-4) -- (0.2,-4.2);
      \draw[thick] (0.2,-4.2) -- (0,-4.3);
    \end{scope}


    \node at (2.7,-5.9) {Wärmeleitband};
    \node at (2.7,-6.9) {$R_{\mathrm{Wärmeleitband}} = \SI{3,617}{\kelvin\per\watt}$};

    \draw[thick] (3.3,-6.4) -- (5.4,-6.4);

    \node at (6.4,-6.4) [style={single arrow, draw}, minimum height=0.5cm, minimum width=1.5cm, shape border rotate=180, thick]{$\frac{\dot{Q}}{4}$}; % Wärmestrom pfeil

    \draw[thick] (0,-6.4) -- (0,-8.4);

    \draw[thick] (0,-8.4) -- (2.1,-8.4);

    \fill (0,-8.4) circle (1.5pt);% knotenpunkt 2

    \begin{scope}[shift={(-1,-8.4)}, rotate=90]
      \draw[thick] (0,-3.1) -- (-0.2,-3.2); % thermal strap resistor 2
      \draw[thick] (-0.2,-3.2) -- (0.2,-3.4);
      \draw[thick] (0.2,-3.4) -- (-0.2,-3.6);
      \draw[thick] (-0.2,-3.6) -- (0.2,-3.8);
      \draw[thick] (0.2,-3.8) -- (-0.2,-4);
      \draw[thick] (-0.2,-4) -- (0.2,-4.2);
      \draw[thick] (0.2,-4.2) -- (0,-4.3);
    \end{scope}

    \node at (2.7,-7.9) {Wärmeleitband};
    \node at (2.7,-8.9) {$R_{\mathrm{Wärmeleitband}} = \SI{3,617}{\kelvin\per\watt}$};

    \draw[thick] (3.3,-8.4) -- (5.4,-8.4);

    \node at (6.4,-8.4) [style={single arrow, draw}, minimum height=0.5cm, minimum width=1.5cm, shape border rotate=180, thick]{$\frac{\dot{Q}}{4}$}; % Wärmestrom pfeil

    \draw[thick] (0,-8.4) -- (0,-9);

    \node[rotate=90] at (0,-9.5) {$\cdots$};

    \draw[->, thick, -{Stealth[length=0.25cm]}] (-1.5,-6) -- node [pos=0,above=1pt] {$g$} (-1.5,-8); % Gravitation
  \end{tikzpicture}
  \caption{\ac{rom} der Thermalen Schnittstelle aus Heatpipe und Wärmeleitbändern. Hier sind nur 2 von 4 Wärmeleitbändern dargestellt.}\label{fig:thermale_schnittstelle}
\end{figure}

\section{PCM}\label{sec:pcm}

Nutzung eines \ac{pcm} mit Fest-Flüssig Übergang ist eine weit verbreitete Lösung in der Luft- und Raumfahrtindustrie um für begrenzte Zeiträume Elektronik in einem akzeptablen
Temperaturbereich zu halten. Auch wenn \ac{pcm} Lösungen generell eine hohe Masse haben, wird das oft aufgrund der ansonsten idealen Eigenschaften inkauf genommen:
Durch die hohe spezifische Schmelzenthalpie, kann rein passiv eine große Wärmemenge, bei einem isothermen Prozess, absorbiert werden. Aufgrund dessen
kann ein von der Umwelt isoliertes \ac{atm} entwickelt werden, das nicht mit stark schwankenden Zuständen der Sonneneinstrahlung und Lufttemperatur
zurecht kommen muss. Auch wenn ein \ac{pcm} mit Flüssig-Gas Übergang meist eine etwa 10-fach höhere Verdampfungsenthalpie hat, wird diese Art
generell nicht verwendet, da der Dichteunterschied zwischen Flüssig- und Gasphase zu extremen Drücken führen würden, falls Wiederverwendbarkeit
verlangt wird und somit ein Druckkörper nötig ist. Alternativ kann die Gasphase auch aus dem Fahrzeug abgelassen werden in einem Prozess der
Vapour Venting genannt wird. Hierbei geht jedoch die Wiederverwendbarkeit verloren, da vor jedem Start die Flüssigphase neu getankt werden muss.
Weiter kann das Vapour Venting trotz der geringen Massenströme zu Momenten führen, die das Fahrzeug destabilisieren; besonders im Überschallbereich
können unintuitive Kräfte durch Interaktionen mit dem Überschallstrom entstehen~\cite{Deere-2011}, die aufwendige \ac{cfd}-Simulationen oder Tests benötigen.
Dementsprechend wird nur ein Fest-Flüssig \ac{pcm} analysiert.

Für die Auswahl eines geeigneten \ac{pcm} sind spezifische Schmelzenthalpie und Schmelztemperatur entscheidend.
Die Wärmeleitfähigkeit ist zwar auch sehr relevant, ist jedoch für alle Materialien zu schlecht und muss durch Lamellen oder ähnliche Wärmetauschende Strukturen verbessert werden,
wobei dabei \ac{pcm} Masse mit Strukturmasse ersetzt wird und somit die Wärmekapazität sinkt. Das Volumen der Wärmeleitenden Struktur welches
\ac{pcm} ersetzt wird Void Fraction genannt, da es gewissermaßen eine Leerstelle im \ac{pcm} bildet, die wie gesagt keine latente Wärmeaufnahme hat. Hier
wird ein Void Fraction von $F = 0.1$ gewählt. Eine Optimierung der Lamellenstruktur kann bei gleich bleibender Masse in einer erhöhten
Wärmeleitfähigkeit resultieren, was jedoch in dieser Arbeit nicht durchgeführt wird. Abbildung \ref{fig:pcm_struktur} zeigt ein Drahtmodell der Struktur.

\begin{table}
  \centering
  \caption{Ampelbewertung für Alkane als \ac{pcm} \cite{NIST}.}\label{tab:pcm_auswahl}
  \label{tab:pcm_alkane_nist}
  \begin{tabular}{>{\raggedright\arraybackslash}m{3.1cm} m{3.1cm} m{3.1cm} m{3.1cm}}
    \toprule[1pt]
    Eigenschaft & n-Hexadecan & n-Octadecan & n-Eicosan \\
    \midrule[0.5pt]

    Schmelzpunkt
      & \cellcolor{bad}\SI{291}{\kelvin}
      & \cellcolor{medium}\SI{301}{\kelvin}
      & \cellcolor{good}\SI{310}{\kelvin} \\

    Schmelzenthalpie
      & \cellcolor{bad}\SI{230400}{\joule\per\kilo\gram}
      & \cellcolor{medium}\SI{239300}{\joule\per\kilo\gram}
      & \cellcolor{good}\SI{240999}{\joule\per\kilo\gram} \\
    \bottomrule[1pt]
  \end{tabular}
\end{table}

Die Auswahltabelle \ref{tab:pcm_auswahl} zeigt die drei gängigsten Organischen Alkane, welche als \ac{pcm} verwendet werden im Vergleich.
Demnach hat n-Eicosan die besten Eigenschaften, mit insbesondere einem perfekten Schmelzpunkt kurz unter den \SI{314,13}{\kelvin} der Senke,
wie in \ref{sec:thermale_schnittstelle} berechnet.
Um die Masse und Dimension des \ac{pcm} zu berechnen wurde das in \ref{lst:pcm_python_pseudo} dargestellte Python-Programm verwendet.
Das \ac{pcm} wird dort als isobar und isotherm angenommen und hat eine unendliche Wärmeleitfähigkeit. Des weiteren befindet es sich
in einer Aluminium-Box mit \SI{1}{\milli\meter} Wanddicke und einem der Void Fraction entsprechenden internen Aluminium Volumenanteil von $F = 0.1$.
Die Dimensionen der Breite und Tiefe der Box wurden gleich gesetzt; die Höhe der Box bildet die zweite Variable.
Kapazitäts- und Massenkonturen abhängig von Seitenlänge und Höhe können in \ref{fig:pcm_waermestrom_vorauslegung} gesehen werden.

Bei einer Flugdauer von \SI{1200}{\second} und einem Wärmestrom von \SI{40}{\watt} ergibt sich eine nötige latente Wärmekapazität von
\SI{48000}{\joule}, eine Seitenlänge der Aluminium-Box von \SI{6,749}{\centi\meter} und eine Gesamtmasse von \SI{346,610}{\gram}.
Da ein Würfel von allen Quadern das größte Volumen-Oberflächenverhältnis hat, sind alle Kanten gleich lang.

\begin{lstlisting}[float, language=Python, caption={Berechnung der Masse und Latenten Wärmekapazität des \ac{pcm} in der pcm.py}, label={lst:pcm_python_pseudo}]
rho_alu = 2700     # aluminium density [kg*m^-3]
rho_pcm = 788      # pcm density [kg*m^-3]
h       = 240998.9 # pcm latent heat [J*kg^-1]
F       = 0.1      # void fraction
t       = 0.001    # wall thickness [m]

def total_mass(L, H): # pcm mass including case and fins
    return (rho_alu * (L**2 * H - (L - 2*t)**2 * (H - 2*t))
            + (F * rho_alu + (1 - F) * rho_pcm) * (L - 2*t)**2 * (H - 2*t)) 

def total_heat(L, H): # pcm latent heat capacity
    #...#
    pcm_heat  = (1 - F) * rho_pcm * (L - 2*t)**2 * (H - 2*t) * h
    return pcm_heat
\end{lstlisting}

\section{Radiator}\label{sec:Radiator}

Bei Radiatoren ist ein hoher Emissions- und niedriger Absorptionsgrad nach Gleichung \ref{eq:radiation} dimensionierend, da die Temperatur den Anforderungen nach limitiert ist
und die Fläche minimiert werden muss, weil diese proportional zu eingehenden Wärmeströmen aus der Umgebung ist, wie etwa die Sonneneinstrahlung oder die Luft, welche auch möglichst gering gehalten werden sollen.

Als Beschichtung wurde AZ-93 der Firma AZ Technology LLC.~\cite{AZ-Technology} ausgewählt. Dabei handelt es sich um eine in der Raumfahrt
weit verbreitete inorganische Farbe mit günstigen Eigenschaften, welche Tabelle \ref{tab:az-93_eigenschaften} entnommen werden können.
Abbildung \ref{fig:radiator_flaeche_leistung} ist eine Visualisierung der Gleichung \ref{eq:radiation} und zeigt Leistungskonturen eines
Radiators mit den Eigenschaften aus Tabelle~\ref{tab:az-93_eigenschaften} je nach Fläche und Temperatur.

\begin{table}

  \centering
  \caption{AZ-93 Spezifikationen~\cite{AZ-Technology}}\label{tab:az-93_eigenschaften}

  \begin{tabular}{ll}

    \toprule[1pt]
    $\varepsilon_{\text{t}}$ & $0.91 \pm 0.02$ \\

    \midrule[0.5pt]
    $\alpha_{\text{s}}$ & $0.15 \pm 0.02$ \\

    \midrule[0.5pt]
    Temperaturbereich  & \SI{-180}{\degreeCelsius} bis \SI{1400}{\degreeCelsius} \\

    \bottomrule[1pt]
  \end{tabular}
\end{table}

Für eine rein radiative Kühlung der Avionik ergibt sich für \SI{40}{\watt} der Avionik und einen spezifischen Wärmestrom der Sonne (Solarer Wärmestrom) von \SI{1}{\kilo\watt\per\meter\squared}
bei einem \SI{50}{\percent} Dutycycle, durch die Rotation der Rakete bzw. der Schattierung des halben Radiators durch die Rakete selbst auf der Sonnenabgewandten Seite, den Eigenschaften aus
\ref{tab:az-93_eigenschaften} und einer Temperatur von $T_\mathrm{senke} = \SI{314,13}{\kelvin}$
nach eine Fläche von \SI{996,163}{\centi\meter\squared}.
Der Solare Wärmestrom wurde für \SI{1}{\meter} über dem Meeresspiegel bei der Sommersonnenwende in Campo Militar de Santa Margarida beim höchsten Sonnenstand
mittels des Online-Tools www.sonnenverlauf.de berechnet, da dort die Demonstrator-Rakete von \ac{blast} starten wird und als Richtwert für \ac{blast} verwendet werden kann.
Die Radiatorleistung ergibt sich demnach zu $\dot{Q}_\mathrm{Radiator} = \SI{47.471}{\watt}$.
In \ref{lst:setup_json} und \ref{lst:radiator_python_pseudo} ist der Programmcode der zur Berechnung verwendet wurde zu sehen.

\begin{lstlisting}[float, language=python, caption={Setup Werte aus der setup.json}, label={lst:setup_json}]
  {
    #...#
    "avionics_power": 40,
    "target_temperature": 36.85,
    "emittance": 0.91,
    "absorptance": 0.15,
    "solar_flux": 1000,
  }
\end{lstlisting}

\begin{lstlisting}[float, language=Python, caption={Berechnung der Radiatorfläche in der radiator.py}, label={lst:radiator_python_pseudo}]
  #...#
  avionics_power = data["avionics_power"]
  e = data["emittance"]
  a = data["absorptance"]
  solar_flux = data["solar_flux"]
  target_temperature = data["target_temperature"]

  def radiator_area(avionics_power, target_temperature, e, a, solar_flux): # radiator area
    return (avionics_power / (e * Stefan_Boltzmann * target_temperature**4 - 0.5 * solar_flux * a))
\end{lstlisting}

\section{PCM-Radiator-Hybrid}\label{sec:pcm_radiator_hybrid}

Eine Hybridlösung wird auch in Erwägung gezogen, um die Masse durch Nutzung eines Radiators zu minimieren, wobei wegen aerodynamischer Aufheizung für kurze Zeit ein PCM gebraucht werden könnte.
Für die Vorauslegung wird die Außenkontur der Rakete von Spitze bis Avionik-Sektion, mit Hilfe der Nußelt-Beziehungen, als Längsangeströmte ebene Platte angesehen,
wie in Abbildung~\ref{fig:rakete_kontour_zeichnung} dargestellt ist.
Um zu wissen, ob hier die Beziehung für laminare oder turbulente Grenzschichten angewandt werden soll, müssen zunächst die Gültigkeitsbereiche der Reynolds- und Prandtlzahl (\ref{eq:prandtl},~\ref{eq:reynolds}) überprüft werden.
Mittels der Nußelt-Beziehung wird der Wärmeübergangskoeffizient $\alpha$ bestimmt und dann in Gleichung~\ref{eq:qdot_recovery} eingesetzt, um auf den spezifischen Wärmestrom zu schließen.

Die Außenstruktur der Rakete besteht aus dem zylindrischen Hüllensegment und einem von-Kármán-Nasenprofil das eine
Spezialform der Haack Serie ist \cite{Stoney-1954}. Die analytische Beschreibung lautet:

\begin{equation*}
  \label{eq:karman_nase}
  x(t) = \frac{R}{\sqrt{\pi}} \cdot \sqrt{
  \cos^{-1}\left(1 - \frac{2t}{L} \right)
  - \frac{1}{2} \cdot \sin\left(2 \cdot \cos^{-1}\left(1 - \frac{2t}{L} \right) \right)
  }
  \quad \text{für } t \in [0, L]
\end{equation*}

Hierbei ist $x(t)$ der Radiusverlauf des rotationssymmetrischen Nasenprofils entlang der Längskoordinate $t$, beginnend an der Spitze $(t=0)$ bis zur Basis $(t=L)$.
Die Gesamtlänge der Nase ist $L = \SI{1250}{\milli\meter}$. Der maximale Radius an der Basis beträgt $R = \SI{125}{\milli\meter}$, und entspricht dem Gesamtdurchmesser der Rakete von $D = \SI{250}{\milli\meter}$.

Die Funktion der Nase wurde mithilfe von einem \ac{cad} Programm skizziert und die Konturlänge zu \SI{1,01}{\meter} vermessen. Wenn der Radiator über den vollständigen Umfang der Rakete
bei einem Durchmesser von $D = \SI{250}{\milli\meter}$ modelliert wird, ist der Radiator eine \SI{12,684}{\centi\meter} lange Sektion.
Daraus folgt eine Konturlänge von \SI{1,074}{\meter} bis zum Mittelpunkt des Radiators, an der Stelle alle lokalen Größen berechnet wurden.

\begin{figure}
  \centering
  \begin{tikzpicture}[rotate border/.style={shape border uses incircle, shape border rotate=#1}, scale=0.8]
    \draw[thick] (3,3) -- (3,-1) -- (9,-1) -- (9,3);
    \draw[thick] (3,3) -- node [midway, above] {Avionik Sektion}  (9,3);
    \draw[thick] (4,3) -- (4,-1); % PCM Lamellen
    \draw[thick] (3,-0.5) -- (4,-0.5);
    \draw[thick] (3,0) -- (4,0);
    \draw[thick] (3,0.5) -- (4,0.5);
    \draw[thick] (3,1) -- (4,1);
    \draw[thick] (3,1.5) -- (4,1.5);
    \draw[thick] (3,2) -- (4,2);
    \draw[thick] (3,2.5) -- (4,2.5);
    \node at (-0.5,2) [style={single arrow, draw}, minimum height=3cm, minimum width=0.5cm, thick]{$\dot{Q}_{\mathrm{Umwelt}}$}; % Wärmestrom pfeil
    \node at (-0.5,0) [style={single arrow, draw}, minimum height=4.5cm, minimum width=1.5cm, shape border rotate=180, thick]{$\dot{Q}_{\mathrm{Radiator}}$}; % Wärmestrom pfeil
    \node at (6.5,1)[style={single arrow, draw}, minimum height=3cm, minimum width=0.5cm, shape border rotate=180, thick]{$\dot{Q}_{\mathrm{Avionik}}$}; % Wärmestrom pfeil
    \draw[->, thick, -{Stealth[length=0.25cm]}] (1,3.75) node [above=1pt] {PCM mit Lamellen} -- (3.5,2.6);
  \end{tikzpicture}
  \caption{PCM Wärmestrom ohne aerodynamische Aufheizung}\label{fig:pcm_waermestrom_diagramm}
\end{figure}

In Abbildung \ref{fig:pcm_waermestrom_diagramm} sieht man eine Schematische Darstellung der Konstruktion und speziell die Wärmeströme
für den Fall, dass das System am Solidus-Punkt im Gleichgewicht steht. Hingegen kann man in Abbildung \ref{fig:pcm_waermestrom_aufheizung_diagramm}
den Zustand sehen, in dem der Umweltwärmestrom durch aerodynamische Aufheizung gestiegen ist und somit das \ac{pcm} anfängt zu schmelzen.
Wegen des \ac{pcm} und der Annahme, dass alle Wärmeleitkoeffizienten unendliche groß sind, wird das System als isotherm modelliert und die Avionik-Sektion als
adiabat.

\begin{figure}
  \centering
  \begin{tikzpicture}[rotate border/.style={shape border uses incircle, shape border rotate=#1}, scale=0.8]
    \draw[thick] (3,3) -- (3,-1) -- (9,-1) -- (9,3);
    \draw[thick] (3,3) -- node [midway, above] {Avionik Sektion}  (9,3);
    \draw[thick] (4,3) -- (4,-1); % PCM lamellen
    \draw[thick] (3,-0.5) -- (4,-0.5);
    \draw[thick] (3,0) -- (4,0);
    \draw[thick] (3,0.5) -- (4,0.5);
    \draw[thick] (3,1) -- (4,1);
    \draw[thick] (3,1.5) -- (4,1.5);
    \draw[thick] (3,2) -- (4,2);
    \draw[thick] (3,2.5) -- (4,2.5);
    \node at (-0.5,2) [style={single arrow, draw}, minimum height=4.5cm, minimum width=1.5cm, thick]{$\dot{Q}_{\mathrm{Umgebung}}$}; % Wärmestrom pfeil
    \node at (-0.5,0) [style={single arrow, draw}, minimum height=4.5cm, minimum width=1.5cm, shape border rotate=180, thick]{$\dot{Q}_{\mathrm{Radiator}}$}; % Wärmestrom pfeil
    \node at (6.5,1)[style={single arrow, draw}, minimum height=3cm, minimum width=0.5cm, shape border rotate=180, thick]{$\dot{Q}_{\mathrm{Avionik}}$}; % Wärmestrom pfeil
  \end{tikzpicture}
  \caption{PCM Wärmestrom bei aerodynamischer Aufheizung}\label{fig:pcm_waermestrom_aufheizung_diagramm}
\end{figure}

\begin{figure}
  \centering
  \begin{tikzpicture}
    \draw[thick] (0,0) arc [start angle=90, end angle=270, x radius=5cm, y radius= 1.5cm]; %nosecone
    \draw[thick] (0,0) -- (3,0); % hülle
    \draw[thick] (0,-3) -- (3,-3); % hülle
    \draw[thick] (-5.75,-1.5) -- (-5.25,-1.5); % maß links
    \draw[thick] (1,0.25) -- (1,0.75); % maß rechts
    \draw[thick] (0,0.5) arc [start angle=90, end angle=180, x radius=5.5cm, y radius= 2cm]; % maß bogen
    \draw[thick] (0,0.5) -- node [near start, above] {Konturlänge} (1,0.5); % maß grade sektion
    \draw[->, thick, -{Stealth[length=0.25cm]}] (-10,0.5) -- node [midway, above] {Luftstrom} (-7,0.5); % free stream pfeile
    \draw[->, thick, -{Stealth[length=0.25cm]}] (-10,-0.5) -- (-7,-0.5); % Strompfeile
    \draw[->, thick, -{Stealth[length=0.25cm]}] (-10,-1.5) -- (-7,-1.5);
    \draw[->, thick, -{Stealth[length=0.25cm]}] (-10,-2.5) -- (-7,-2.5);
    \draw[->, thick, -{Stealth[length=0.25cm]}] (-10,-3.5) -- (-7,-3.5);
    \draw[thick] (0,0) -- (0,-2.5); % casing wall
    \draw[thick] (2,0) -- (2,-2.5); % casing wall
    \draw[thick] (0,-2.5) -- (2,-2.5); % casing bottom
    \node at (1,-1.5) [style={single arrow, draw}, minimum height=0.5cm, minimum width=1.5cm, rotate=90, thick]{$\dot{Q}_{\mathrm{Avionik}}$}; % Wärmestrom pfeil
  \end{tikzpicture}
  \caption{Kontourlänge vom Staupunkt der Rakete bis zum Mittelpunkt des Radiators}\label{fig:rakete_kontour_zeichnung}
\end{figure}


Die Software zur Berechnung aller Bilanzgleichungen für Dimensionen und Massen besteht aus einer Reihe an Python Programmen und Datenstrukturen.
Das Programm main.py ist das Hauptprogramm das alle Unterprogramme in der richtigen Reihenfolge aufruft. Zuerst die radiator.py zur Berechnung
der Radiator Dimension mithilfe der Randbedingungen aus der setup.json. Danach wird die hybrid.py aufgerufen um die Aerodynamischen Wärmeströme
mittels der Nußelt-Beziehung zu bestimmen. Die Ergebnisse werden anschließend in die pcm.py geladen, um abhängig von der Radiatorfläche
und den Wärmeströmen die Kapazität und Masse des \ac{pcm} zu bestimmen.
In Abbildung~\ref{fig:dimensionierung_ablauf} sieht man schematisch wie die Dimensionierung in der Software abläuft.

\begin{figure}
  \centering
  \begin{tikzpicture}[
    sibling distance=10em,
    every node/.style = {
      shape=rectangle,
      rounded corners,
      draw,
      align=center,
      minimum width=3cm
    },
    edge from parent/.style = {
      draw,
      ->,
      -{Stealth[length=0.25cm]},
      thick
    },
    arrow/.style = {
      ->,
      -{Stealth[length=0.25cm]},
      thick
    }
  ]

    % Top nodes
    \node (avionik) at (-2.1, 4) {Avionik Wärmestrom};
    \node (sonne)   at ( 2.1, 4) {Solarer Wärmestrom};

    % Radiator Fläche centered below
    \node (radiator) at (0, 2) {Radiator Fläche};

    % Children of Radiator
    \node (breite) at (-2.4, 0) {PCM Breite};
    \node (aerodynamisch) at (2.4, 0) {Aerodynamischer Wärmestrom};

    % PCM Kapazität as a separate node (not a child directly)
    \node (kapazitaet) at (2.1, -2) {PCM Kapazität};

    % PCM Höhe node below the center of breite and kapazitaet
    \node (hoehe) at (0, -4) {PCM Dicke};
    \node (gewicht) at (0, -6) {PCM Gewicht};

    % Arrows
    \draw[arrow] (avionik) -- (radiator);
    \draw[arrow] (sonne) -- (radiator);
    \draw[arrow] (radiator) -- (breite);
    \draw[arrow] (radiator) -- (aerodynamisch);
    \draw[arrow] (aerodynamisch) -- (kapazitaet);
    \draw[arrow] (breite) -- (hoehe);
    \draw[arrow] (kapazitaet) -- (hoehe);
    \draw[arrow] (hoehe) -- (gewicht);

  \end{tikzpicture}
  \caption{Ablauf der Dimensionierung in der Vorauslegungs-Software}\label{fig:dimensionierung_ablauf}
\end{figure}

Wie in Abbildung \ref{fig:re_pr_flugsimulation} dargestellt, liegt die Prandtl-Zahl im Gültigkeitsbereich sowohl für die turbulente als auch
für die laminare Grenzschicht. Die Reynolds-Zahl überschreitet jedoch zeitweise mit Werten bis zu \SI{2.4e7} die Gültigkeitsbereiche.
Aufgrund fehlender alternativer analytischer Methoden wurde dennoch die Nußelt-Beziehung \ref{eq:nusselt_turbulent} für turbulente Grenzschichten angewendet.

\begin{figure}
  \centering
  \includegraphics[width=\linewidth]{../../Code/re_pr_during_flight.pdf}
  \caption{Reynolds- und Prandtlzahl während kritischer Phase im Flug}\label{fig:re_pr_flugsimulation}
\end{figure}

Das Ergebnis der Berechnung ist in Abbildung \ref{fig:pcm_waermestrom_vorauslegung} zu sehen, mit der Radiatorleistung $\dot{Q}_\mathrm{Radiator} = \SI{47.471}{\watt}$
aus \ref{sec:Radiator} und dem Wärmestrom aus der Umwelt $\dot{Q}_\mathrm{Umwelt}$, der aus Sonneneinstrahlung wie in \ref{sec:Radiator} modelliert
und der aerodynamischen Aufheizung besteht.
Der Wärmestrom $\dot{Q}_\mathrm{Rein}$ ist die Summe aus Avionik Wärmestrom $\dot{Q}_\mathrm{Avionik}$ und dem Umwelt Wärmestrom $\dot{Q}_\mathrm{Umwelt}$.

Erkennbar ist, dass mit bis zu \SI{20}{\kilo\watt} ein extrem hoher Wärmestrom durch die aerodynamisch Aufheizung entsteht.
Auch wenn dieser nur etwa \SI{100}{\second} andauert, ist er ausreichend um die notwendige Masse des \ac{pcm} (inklusive der Aluminium Struktur)
auf \SI{4,256}{\kilo\gram} zu erhöhen. Abgesehen von der höheren notwendigen Kapazität von \SI{626817.571}{\joule} führen auch zusätzlich geometrische
Verluste zu der erhöhten Masse, da das Aspektverhältnis aufgrund der Einschränkung durch die Radiatorfläche weit von der idealen Würfelform entfernt ist.

\begin{figure}
  \centering
  \includegraphics[width=\linewidth]{../../Code/pcm_radiator_hybrid_heatflux_nosim.pdf}
  \caption{PCM Wärmeströme während dem Flug}\label{fig:pcm_waermestrom_vorauslegung}
\end{figure}