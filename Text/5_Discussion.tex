\chapter{Discussion and conclusions}
\label{chap:conclusion}
random zeug

Vor der Implementierung des \ac{pcm} in Hardware, sollten die Eigenschaften des vorhandenen n-Eicosan nochmal analysiert und die Ergebnisse überprüft werden.

Wenn man bessere Finnen konstruiert braucht man feineres mesh um gradienten in den wänden zu sehen


Die Vorauslegungwurde mit folgenden Werten durchgeführt:\\
- Isotherm auf: \SI{38}{\celsius}\\
- Avionik Abwärme: \SI{40}{W}\\
- \SI{1}{m} Kontourlänge\\
- Radiator Emissionsgrad: \SI{0,91}{} (AZ-93)\\
- Radiator Absorptionsgrad: \SI{0,15}{} (AZ-93)\\
- Icosane PCM\\
- Trajektoriensimulation\\
- \SI{1}{\kilo\watt\per\meter\squared} mit 50\% dutycycle durch Rotation der Rakete\\
Zu beachten ist, dass die Radiatorleistung konstant bleibt, da das System als isotherm mit einer
infinitesimalen Temperaturerhöhung über den Schmelzpunkt hinweg angenommen wird.\\
Als nächstes sieht man die Flugdaten

\section{Discussion about including pictures}
