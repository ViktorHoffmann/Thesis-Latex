\chapter*{Appendix}
\label{chapter:Appendix}
\pagestyle{Appendix}
\addcontentsline{toc}{chapter}{Appendix}

\section*{Appendix A: Simulationsergebnisse}\label{Anh:simulation}

\begin{figure}[H]
    \centering

    \begin{subfigure}{\textwidth}
        \centering
        \includegraphics[height=0.23\textheight]{ansyspost/airflow/maxQminus10-temperature-contour.png}
        \caption{maxQ -\SI{10}{\second}}
        \label{fig:maxQminus10_temp_contour}
    \end{subfigure}

    \begin{subfigure}{\textwidth}
        \centering
        \includegraphics[height=0.23\textheight]{ansyspost/airflow/maxQplus10-temperature-contour.png}
        \caption{maxQ +\SI{10}{\second}}
        \label{fig:maxQplus10_temp_contour}
    \end{subfigure}

    \begin{subfigure}{\textwidth}
        \centering
        \includegraphics[height=0.23\textheight]{ansyspost/airflow/maxQplus20-temperature-contour.png}
        \caption{maxQ +\SI{20}{\second}}
        \label{fig:maxQplus20_temp_contour}
    \end{subfigure}

    \caption{Statische Temperaturkontur der Luft}
    \label{fig:airflow_temp_contour_continued}
\end{figure}

\begin{figure}[H]
    \centering

    \begin{subfigure}{\textwidth}
        \centering
        \includegraphics[height=0.23\textheight]{ansyspost/airflow/maxQminus10-mach-contour.png}
        \caption{maxQ -\SI{10}{\second}}
        \label{fig:maxQminus10_mach_contour}
    \end{subfigure}

    \begin{subfigure}{\textwidth}
        \centering
        \includegraphics[height=0.23\textheight]{ansyspost/airflow/maxQplus10-mach-contour.png}
        \caption{maxQ +\SI{10}{\second}}
        \label{fig:maxQplus10_mach_contour}
    \end{subfigure}

    \begin{subfigure}{\textwidth}
        \centering
        \includegraphics[height=0.23\textheight]{ansyspost/airflow/maxQplus20-mach-contour.png}
        \caption{maxQ +\SI{20}{\second}}
        \label{fig:maxQplus20_mach_contour}
    \end{subfigure}

    \caption{Machzahlkontur der Luft}
    \label{fig:airflow_mach_contour_continued}
\end{figure}

\begin{figure}[H]
    \centering

    \begin{subfigure}[t]{0.14\textwidth}
        \centering
        \raisebox{1\height}{\includegraphics[height=0.2\textheight]{ansyspost/pcm/vector-legend.png}}
    \end{subfigure}%
    \hspace{2mm}% extra space between legend and first image
    \begin{subfigure}[t]{0.2\textwidth}
        \centering
        \includegraphics[height=0.7\textheight]{ansyspost/pcm/velocity-vector-300.png}
        \caption{\SI{300}{\second}}\label{fig:velocity_vector_300}
    \end{subfigure}%
    \begin{subfigure}[t]{0.2\textwidth}
        \centering
        \includegraphics[height=0.7\textheight]{ansyspost/pcm/velocity-vector-600.png}
        \caption{\SI{600}{\second}}\label{fig:velocity_vector_600}
    \end{subfigure}%
    \begin{subfigure}[t]{0.2\textwidth}
        \centering
        \includegraphics[height=0.7\textheight]{ansyspost/pcm/velocity-vector-900.png}
        \caption{\SI{900}{\second}}\label{fig:velocity_vector_900}
    \end{subfigure}%
    \begin{subfigure}[t]{0.2\textwidth}
        \centering
        \includegraphics[height=0.7\textheight]{ansyspost/pcm/velocity-vector-1200.png}
        \caption{\SI{1200}{\second}}\label{fig:velocity_vector_1200}
    \end{subfigure}
    \caption{Konturen der statischen Temperatur. Die Legende bezieht sich auf~\ref{fig:temperatur_1200}}\label{fig:pcm_static_temperature_kontur}
\end{figure}

\section*{Appendix B: bla}
\label{Anh:bla2}