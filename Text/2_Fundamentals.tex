 \chapter{Grundlagen}
\label{chap:Grundlagen}			% durch \label{} später auf Element verweisen

% Bücher, aber auch Paper etc. 

\section{Sensible Wärmespeicher}
\label{sec:sensiblewaermespeicher}
Unter sensibler Wärme versteht man die Eigenschaft von 
\begin{equation}
c = \frac{\Delta Q}{m \cdot \Delta T}
\end{equation}
Diese Thermodynamische Eigenschaft lässt sich somit leicht nutzen, um für einen begrenzten Zeitraum die Temperatur gewissermaßen zu dämpfen. Da in der Realität Elektronik keine Wärme produzieren kann, ohne auch Masse im System zu haben, hat jede Avionik inherent eine sogenannte Thermale Masse, welche sich bei Nutzung der sensiblen Wärme leicht durch hinzufügen von Heatsinks erhöhen lässt.\\
Der eine, und auch größte, Nachteil von dieser Art an Thermal-Management ist, dass die Masse des Systems proportional zur Wärmekapazität steigt.

\section{Latente Wärmespeicher}
\label{sec:latentewaermespeicher}

\section{Radiator}
\label{sec:radiator}

\section{Hybrid Lösung}
\label{sec:hybridloesung}

\newpage
