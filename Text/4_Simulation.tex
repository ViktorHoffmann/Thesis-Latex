\chapter{Simulation}\label{chap:Simulation}

Um die Ergebnisse der Vorauslegung zu verifizieren wurde mithilfe ANSYS Fluent sowohl das Verhalten des \ac{pcm}, als auch die
aerodynamische Aufheizung simuliert.

\section{PCM}\label{sec:sim_pcm}

Für die Simulation des \ac{pcm} wurde die in \ref{fig:pcm_struktur} dargestellte Struktur stark vereinfacht,
um trotz mangelnder Rechenressourcen gelöst werden zu können. Zuerst wurde das \ac{pcm} in der Symmetrieebene
zu einem zweidimensionalen Problem vereinfacht. Im nächsten Schritt wurde nur die mittlere Zelle aus der Ebene unter der Annahme,
dass das System symmetrisch ist ausgewählt. Im letzten Schritt wurde die Zelle nochmal aufgrund von Symmetrie gespalten.

Anschließend wurde in ANSYS Mechanical das Mesh vollständig aus Tetraeder-Elementen erzeugt, wobei die Zellgröße so gewählt
wurde, dass die Aluminium-Wände 1-2 Zellen Tiefe haben. In dem Mesh \ref{fig:pcm_mesh} ist das \ac{pcm} in schwarz und das Aluminium
in rot dargestellt. Jeweils an der linken und rechten Kante, wurde aufgrund der anliegenden Zelle bzw. Spiegelung der Zelle eine
Symmetrie Randbedingung gewählt. Die untere Kante wurde als Wärmequelle angelegt und die obere Kante als adiabate Wand.

Die Wärmequelle ergibt sich aus der Seitenfläche der \ac{pcm} Struktur, bestimmt in \ref{sec:pcm}, und dem Avionik Wärmestrom zu
$\frac{\SI{40}{\watt}}{\left(\SI{6,749}{\centi\meter}\right)^2} = \SI{8782}{\watt\per\meter\squared}$.
Da ANSYS Fluent bei zweidimensionalen Simulationen eine Referenztiefe von \SI{1}{m} verwendet, konnte der spezifische Wärmestrom
der Quelle direkt verwendet werden.

Die Thermodynamischen Eigenschaften von n-Eicosan sind aufgeführt in Tabelle \ref{tab:eicosane_data}.
Das temperaturabhängige Verhalten der spezifischen Wärmekapazität kann den Graphen \ref{fig:pcm_effective_cp} und \ref{fig:pcm_sensible_cp}
entnommen werden.

\begin{figure}[!htb]
    \centering
    \begin{subfigure}[t]{0.7\textwidth}
        \centering
        \includegraphics[height=9cm]{ansyspost/pcm/40WPCM_struktur.png}
        \caption{\ac{pcm} Struktur}\label{fig:pcm_struktur}
    \end{subfigure}
    \hfill
    \begin{subfigure}[t]{0.15\textwidth}
        \centering
        \includegraphics[height=9cm]{ansyspost/pcm/2DPCM_mesh.png}
        \caption{\ac{pcm} Mesh}\label{fig:pcm_mesh}
    \end{subfigure}
    \caption{\ac{pcm} Struktur und vereinfachtes Mesh}\label{fig:pcm_geometrien}
\end{figure}

\begin{table}[H]

  \centering
  \caption{Stoffdaten für n-Eicosan}\label{tab:eicosane_data}

  \begin{tabular}{lll}

    \toprule[1pt]
    Solidus Temperatur & $T_{\text{solidus}}$ & \SI{309}{\kelvin}~\cite{NIST} \\

    \midrule[0.5pt]
    Liquidus Temperatur & $T_{\text{liquidus}}$ & \SI{311}{\kelvin}~\cite{NIST} \\

    \midrule[0.5pt]
    Spezifische Wärmekapazität bei\\konstantem Druck der\\Flüssigphase & $c_{p,\text{liquid}}$ & \SI{2350.05}{\joule\per\kilogram\per\kelvin}~\cite{NIST} \\

    \midrule[0.5pt]
    Spezifische Wärmekapazität bei\\konstantem Druck der\\Feststoffphase & $c_{p,\text{solid}}$ & \SI{2132.4}{\joule\per\kilogram\per\kelvin}~\cite{NIST} \\

    \midrule[0.5pt]
    Dichte der Flüssigphase & $\rho_{\text{solid}}$ & \SI{910}{\kilogram\per\cubic\meter}~\cite{Nazarychev-2022} \\

    \midrule[0.5pt]
    Dichte der Feststoffphase & $\rho_{\text{liquid}}$ & \SI{769}{\kilogram\per\cubic\meter}~\cite{Nazarychev-2022} \\

    \midrule[0.5pt]
    Wärmeleitfähigkeit der Flüssigphase & $\lambda_{\text{liquid}}$ & \SI{0.1505}{\watt\per\meter\per\kelvin}~\cite{Benbrika-2020} \\

    \midrule[0.5pt]
    Wärmeleitfähigkeit der Feststoffphase & $\lambda_{\text{solid}}$ & \SI{0.4248}{\watt\per\meter\per\kelvin}~\cite{Stryker-1990} \\

    \midrule[0.5pt]
    Wärmeausdehnungskoeffizient & $\beta$ & \SI{0.0009}{\per\kelvin}~\cite{Benbrika-2020} \\

    \midrule[0.5pt]
    Spezifische Schmelzenthalpie & $h_{\text{fus}}$ & \SI{240998.86}{\joule\per\kilogram}~\cite{NIST} \\

    \bottomrule[1pt]
  \end{tabular}
\end{table}

\begin{figure}[H]
  \centering
  \includegraphics[width=\linewidth]{../../Code/eicosane_cpvst_total.pdf}
  \caption{Effektive spezifische Wärmekapazität von n-Eicosan}\label{fig:pcm_effective_cp}
\end{figure}

\begin{figure}[H]
  \centering
  \includegraphics[width=\linewidth]{../../Code/eicosane_cpvst_sensible.pdf}
  \caption{Sensible spezifische Wärmekapazität von n-Eicosan}\label{fig:pcm_sensible_cp}
\end{figure}

Die Simulation wurde mit dem Pressure-Based Solver~\cite{akamcae-udf} als transiente Simulation über 120000 Zeitschritte mit einer Zeitschrittgröße von \SI{0,01}{\second} durchgeführt
um die vollständige Flugdauer mit \SI{1200}{\second} zu simulieren.
Des weiteren wurde das Energiemodell eingeschaltet~\cite{akamcae-udf}, das Viskositätsmodell als Laminar angenommen~\cite{akamcae-udf} und das Phasenwechselmodell aktiviert~\cite{akamcae-udf}.
Neben den bereits erläuterten \ac{pcm} Eigenschaften wurde für das Aluminium eine Dichte von $\rho = \SI{2719}{\kilogram\per\meter\cubed}$,
eine spezifische Wärmekapazität von $c_p = \SI{871}{\joule\per\kilogram\kelvin}$ und eine Wärmeleitfähigkeit von $\lambda = \SI{202.4}{\watt\per\meter\kelvin}$
eingestellt.

Abbildung \ref{fig:approximierte_beschleunigung} zeigt das Beschleunigungsprofil, welches in der Simulation verwendet wurde. Zu beachten
ist, dass Beschleunigungsspitzen durch den Fallschirm, wie sie in~\ref{fig:acceleration_over_time} gesehen
werden können, ignoriert werden, da diese durch mangelhafte Genauigkeit der Fallschirm Modellierung resultieren.

\begin{figure}
  \centering
  \includegraphics[width=\linewidth]{../../Code/approximate_acceleration_over_time.pdf}
  \caption{Approximiertes Beschleunigungsprofil}\label{fig:approximierte_beschleunigung}
\end{figure}

Da die ANSYS Fluent Transient Table Funktion (Native Funktion für transiente Randbedingungen mit Profilen) keine transiente Gravitation
unterstützt, wurde diese und die globale Beschleunigung deaktiviert.
Stattdessen wurde die Beschleunigung über den Quellterm der Boussinesq-Approximation in der \ac{udf} implementiert. Die Funktion ist in
\ref{lst:udf_bossinesque} zu sehen.

Als Scheme wurde SIMPLE verwendet~\cite{akamcae-udf}, für die Gradienten Least Squares Cell Based~\cite{akamcae-udf}, für Druck Second Order~\cite{akamcae-udf} und für Impuls und Energie
Second Order Upwind~\cite{akamcae-udf}. Die Unterrelaxationsfaktoren wurden durch experimentelle Ermittlung anhand der Residuen zu 0,3 für Druck, 1 für Dichte
und Körperkräfte, 0,5 für Impuls und 0,9 für sowohl Flüssigkeitsanteil als auch Energie gewählt.

\begin{figure}
    \centering

    % Left figure
    \begin{minipage}[t]{0.485\textwidth}
        \centering
        \setlength{\tabcolsep}{1pt} % reduce subfigure spacing
        % Legend vertically centered & with extra space to right
        \begin{subfigure}[t]{0.16\textwidth}
            \centering
            \raisebox{0.7\height}{\includegraphics[height=0.2\textheight]{ansyspost/pcm/liquid-fraction-legend.png}}
        \end{subfigure}%
        \hspace{2mm}% extra space between legend and first image
        \begin{subfigure}[t]{0.2\textwidth}
            \centering
            \includegraphics[height=0.5\textheight]{ansyspost/pcm/liquid-fraction-300.png}
            \caption{\SI{300}{\second}}\label{fig:liquid_fraction_300}
        \end{subfigure}%
        \begin{subfigure}[t]{0.2\textwidth}
            \centering
            \includegraphics[height=0.5\textheight]{ansyspost/pcm/liquid-fraction-600.png}
            \caption{\SI{600}{\second}}\label{fig:liquid_fraction_600}
        \end{subfigure}%
        \begin{subfigure}[t]{0.2\textwidth}
            \centering
            \includegraphics[height=0.5\textheight]{ansyspost/pcm/liquid-fraction-900.png}
            \caption{\SI{900}{\second}}\label{fig:liquid_fraction_900}
        \end{subfigure}%
        \begin{subfigure}[t]{0.2\textwidth}
            \centering
            \includegraphics[height=0.5\textheight]{ansyspost/pcm/liquid-fraction-1200.png}
            \caption{\SI{1200}{\second}}\label{fig:liquid_fraction_1200}
        \end{subfigure}
        \caption{Flüssigkeitsanteil Konturen. Die Legende bezieht sich auf~\ref{fig:liquid_fraction_1200}}
        \label{fig:liquid_frac_kontur}
    \end{minipage}
    \hspace{2mm} % small horizontal space
    % Right figure
    \begin{minipage}[t]{0.485\textwidth}
        \centering
        \begin{subfigure}[t]{0.16\textwidth}
            \centering
            \raisebox{0.7\height}{\includegraphics[height=0.2\textheight]{ansyspost/pcm/temperature-legend.png}}
        \end{subfigure}%
        \hspace{2mm}% extra space between legend and first image
        \begin{subfigure}[t]{0.2\textwidth}
            \centering
            \includegraphics[height=0.5\textheight]{ansyspost/pcm/static-temperature-300.png}
            \caption{\SI{300}{\second}}\label{fig:temperatur_300}
        \end{subfigure}%
        \begin{subfigure}[t]{0.2\textwidth}
            \centering
            \includegraphics[height=0.5\textheight]{ansyspost/pcm/static-temperature-600.png}
            \caption{\SI{600}{\second}}\label{fig:temperatur_600}
        \end{subfigure}%
        \begin{subfigure}[t]{0.2\textwidth}
            \centering
            \includegraphics[height=0.5\textheight]{ansyspost/pcm/static-temperature-900.png}
            \caption{\SI{900}{\second}}\label{fig:temperatur_900}
        \end{subfigure}%
        \begin{subfigure}[t]{0.2\textwidth}
            \centering
            \includegraphics[height=0.5\textheight]{ansyspost/pcm/static-temperature-1200.png}
            \caption{\SI{1200}{\second}}\label{fig:temperatur_1200}
        \end{subfigure}
        \caption{Konturen der statischen Temperatur. Die Legende bezieht sich auf~\ref{fig:temperatur_1200}}
        \label{fig:static_temperature_kontur}
    \end{minipage}

\end{figure}

In Abbildung \ref{fig:static_temperature_kontur} und \ref{fig:liquid_frac_kontur} kann man jeweils die Lösung des Flüssigkeitsanteils
und der statischen Temperatur zu mehreren Zeitschritten sehen. Man kann dort deutlich erkennen, wie das \ac{pcm} von der Wärmequelle aus
schmilzt. Besonders an der Aluminiumlamelle bildet sich eine beschleunigte Konvektion die jedoch nach unten fließt und durch das aufsteigende
\ac{pcm} in der Mitte der Zelle angetrieben wird. Im Vektorfeld \ref{fig:pcm_vectoren_stitched} kann man den dadurch entstandenen Wirbel sehen.

Besonders interessant ist, dass wie in \ref{fig:static_temperature_kontur} zu erkennen ist, die Temperatur an der Quelle auf bis zu \SI{336}{\kelvin}
steigt. Demnach würde mittels der Thermalen Schnittstelle aus \ref{sec:thermale_schnittstelle} die Gehäusetemperatur der Avionik-Bauteile mit
$T_C = \SI{384,17}{\kelvin}$ über die zuverlässige Temperatur steigen.

\begin{lstlisting}[language=C, float, caption={Boussinesq-Approximation des Auftriebs im \ac{pcm} in der \ac{udf} eicosane.c}, label={lst:udf_bossinesque}]
//Y-momentum source
DEFINE_SOURCE(Boussinesq_momentum_source,cell,thread,dS,eqn)
{
	double Temp, source, acc;
	Temp=C_T(cell,thread);

	double t = CURRENT_TIME;

	if (t < 20)
		acc = 34.81;
	else if (t < 50)
		acc = 109.81;
	else if (t < 150)
		acc = 19.62;
	else
		acc = 9.81;

	source=-Rol_pcm*acc*TEC*(Temp-Tr);  //negative for -Y down
	dS[eqn]=-Rol_pcm*acc*TEC; 			//negative for -Y down
	return source;
}
\end{lstlisting}

\begin{figure}
  \centering
  \includegraphics[height=0.4\textheight]{ansyspost/pcm/velocity-vector-close-stitched-900.png}
  \caption{Geschwindigkeitsvektoren der Konvektionswirbel einer, durch Nachbearbeitung, vervollständigten Zelle
  bei \SI{900}{\second}. Darstellung der weiteren Zeitschritte ist in~\ref{fig:pcm_static_temperature_kontur} zu finden.}\label{fig:pcm_vectoren_stitched}
\end{figure}

\section{Aerodynamische Aufheizung}\label{sec:sim_aerodynamisch}
Die Simulation der aerodynamischen Aufheizung wurde als stationäre Simulation mit dem Density-Based solver durchgeführt. Hierbei wurde
aufgrund der Rotationssymmetrie der Rakete im relevanten Bereich oberhalb der Finnen wieder eine zweidimensionale Simulation durchgeführt.
Als Viskositätsmodell wurde SST k-$\omega$~\cite{Irving-2021} gewählt und das Energiemodell aktiviert.
Die Luft wurde als Ideales Gas mit einer spezifischen Wärmekapazität von \SI{1006.43}{\joule\per\kilogram\kelvin},
einer Wärmeleitfähigkeit von \SI{0.0242}{\watt\per\meter\kelvin}, einer dynamischen Viskosität von \SI{1.7894e-5}{\kilogram\per\meter\second}
und einer molekularen Masse von \SI{28,966}{\kilogram\per\kilo\mole} modelliert.

Die Außenstruktur der Rakete besteht aus dem zylindrischen Hüllensegment und einem von-Kármán-Nasenprofil das analytisch
beschrieben wird:

\begin{equation*}
  \label{eq:karman_nase}
  x(t) = \frac{R}{\sqrt{\pi}} \cdot \sqrt{
  \cos^{-1}\left(1 - \frac{2t}{L} \right)
  - \frac{1}{2} \cdot \sin\left(2 \cdot \cos^{-1}\left(1 - \frac{2t}{L} \right) \right)
  }
  \quad \text{für } t \in [0, L]
\end{equation*}

Hierbei ist $x(t)$ der Radiusverlauf des rotationssymmetrischen Nasenprofils entlang der Längskoordinate $t$, beginnend an der Spitze $(t=0)$ bis zur Basis $(t=L)$.
Der maximale Radius an der Basis beträgt $R = \SI{125}{\milli\meter}$ und die Gesamtlänge der Nase $L = \SI{1250}{\milli\meter}$.
Dies entspricht dem Gesamtdurchmesser der Rakete von $D = \SI{250}{\milli\meter}$.

Das Mesh der Domäne wurde in ANSYS Mechanical vollständig aus Tetraeder-Elementen erstellt und ist samt Randbedingungen in \ref{fig:aussenstroemung_mesh} dargestellt.
Um die Grenzschicht-Anforderung aus \ref{eq:yplus} mit $y^+ \leq 1$ zu erfüllen, wurden Inflationsschichten an der Raketenwand eingefügt, die in \ref{fig:aussenstroemung_mesh_inflationlayers}
zu sehen sind. Die Höhe der ersten Schicht wurde experimentell erhalten \cite{Anderson-2017}

Die Vorauslegungwurde mit folgenden Werten durchgeführt:\\
- Isotherm auf: \SI{38}{\celsius}\\
- Avionik Abwärme: \SI{40}{W}\\
- \SI{1}{m} Kontourlänge\\
- Radiator Emissionsgrad: \SI{0,91}{} (AZ-93)\\
- Radiator Absorptionsgrad: \SI{0,15}{} (AZ-93)\\
- Icosane PCM\\
- Trajektoriensimulation\\
- \SI{1}{\kilo\watt\per\meter\squared} mit 50\% dutycycle durch Rotation der Rakete\\
Zu beachten ist, dass die Radiatorleistung konstant bleibt, da das System als isotherm mit einer
infinitesimalen Temperaturerhöhung über den Schmelzpunkt hinweg angenommen wird.\\
Als nächstes sieht man die Flugdaten

Speziell für die Strömungssimulationen welche keine Koppelung mit Festkörpern haben, wurde der Density-Based Solver ausgewählt und die
Simulation als 2D Steady State durchgeführt. Das Energiemodell wurde aktiviert und für das Viskositätsmodell~\cite{Irving-2021}

Die Umströmungssimulationen der Rakete wurden an \ac{maxq} orientiert, da es als Richtwert für Aerodynamische Aufheizung genommen werden kann.
Desweiteren ist der Wert unanhängig von der Vorauslegung, wodurch Ungenauigkeiten von dort getroffenen Annahmen vermieden werden.\\


als nächstes habe ich geschaut wo der maximale dynamische Druck erreicht wurde in der Vorauslegung. Die korrespondierenden Werte des Flugzustandes
habe ich dann als Boundery Conditions in der \ac{cfd}~Simulation genommen.
Um zu verifizieren, dass dort auch die maximale Aufheizung stattfindet, habe ich 1 Sekunden vorher und nachder
im Flug die BC's auch verwendet und einen Vergleich gezogen.\\
Maximaler dynamischer Druck: 112901.25708461029 Pa at 28.691 s\\
Entsprechender Flugzustand: 10244.138 m, 750.704 m/s, -51.587°C, 254.783 hPa mit entsprechender Luft Dichte \SI{0.4006}{kg/m^3}\\
Flugzustand bei 18.691 s \ac{maxq} - $\SI{10}{\second}$: 4274.387 m, 461.355 m/s, -12.784, 594.935 hPa mit entsprechender Luft Dichte \SI{0.7960}{kg/m^3}\\
Flugzustand bei 38.691 s \ac{maxq} + $\SI{10}{\second}$: 19758.652 m, 1189.968 m/s, -56.5°C, 56.93 hPa mit entsprechender Luft Dichte \SI{0.0915}{kg/m^3}\\
Flugzustand bei 48.7 s \ac{maxq} + $\SI{20}{\second}$: 32439.616 m, 1393.377 m/s, -43.269°C, 8.136 hPa mit entsprechender Luft Dichte \SI{0.01233001}{kg/m^3}\\
Da wie in~\ref{fig:spezifischer_waermestrom_maxQ_simulationen} zu sehen ist, der Zeitpunkt des maximalen dynamischen Druckes nicht im größten spezifischen
Wärmestrom resultiert, wurde mit der Simulation die den höheren spezifischen Wärmestrom ergeben hat, eine Lösungsfortsetzung durchgeführt um das Maximum zu finden.\\

\begin{figure}[H]
  \centering
  \includegraphics[width=\linewidth]{ansyspost/airflow/mesh_all.png}
  \caption{Darstellung der Außensströmungssimulation mit Meshstruktur in grau, velocity inlet in blau, pressure outlet in rot, Symmetrien in gelb und Partitionen der parallelisierung in lila}\label{fig:aussenstroemung_mesh}
\end{figure}

\begin{figure}[H]
  \centering
  \includegraphics[width=\linewidth]{ansyspost/airflow/mesh_inflation.png}
  \caption{Schichtaufdickungen des Mesh an der Rakete}\label{fig:aussenstroemung_mesh_inflationlayers}
\end{figure}

\begin{figure}[H]
  \centering
  \includegraphics[width=\linewidth]{../../Code/maxQ_compare_heatflux.pdf}
  \caption{Spezifischer Wärmestrom an der Außenhaut bei maximalem dynamischen Druck, sowie \SI{10}{s} davor, danach und \SI{20}{s} danach}\label{fig:spezifischer_waermestrom_maxQ_simulationen}
\end{figure}

\begin{figure}[H]
  \centering
  \includegraphics[width=\linewidth]{../../Code/maxQ_compare_yplus.pdf}
  \caption{y+ Wert an der Außenhaut bei \ac{maxq}, sowie \SI{10}{s} davor, danach und \SI{20}{s} danach}\label{fig:yplus_maxQ_simulationen}
\end{figure}

\begin{figure}[H]
  \centering
  \includegraphics[width=\linewidth]{../../Code/pcm_radiator_hybrid_heatflux_with_sim.pdf}
  \caption{PCM Wärmestrom während Flug mit Simulationsergebnissen und Fit Kurve}\label{fig:pcm_waermestrom_sim}
\end{figure}

Fitted Gaussian parameters:
a=12454028.32, b=32.87, c=550.50, d=-12446646.16

\begin{figure}[H]
    \centering

    \begin{subfigure}{\textwidth}
        \centering
        \includegraphics[height=0.23\textheight]{ansyspost/airflow/maxQ-temperature-contour.png}
        \caption{Statische Temperaturkontur der Luft}
        \label{fig:maxQ_temp_contour}
    \end{subfigure}

    \begin{subfigure}{\textwidth}
        \centering
        \includegraphics[height=0.23\textheight]{ansyspost/airflow/maxQ-mach-contour.png}
        \caption{Machzahlkontur der Luft}
        \label{fig:maxQ_mach_contour}
    \end{subfigure}

    \caption{\texorpdfstring{\ac{maxq}}{max Q} Konturen der Luft}
    \label{fig:maxQ_konturen}
\end{figure}